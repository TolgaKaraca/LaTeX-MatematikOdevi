\documentclass[a4paper,12pt,twocolumn]{article}

\usepackage{systeme}
\usepackage{tikz}

\begin{document}
\title{\Large{\textbf{Ödev}}}
\author{Tolga Karaca}
\date{4 Nisan 2020}
\maketitle

\section{Sorular}

\textbf{Soru 1-} $a$,$b$ ve $c$ birer tam sayıdır.$$ab=12,bc=18$$ olduğuna göre $a+b+c$'nin alabileceği en büyük değer ile en küçük değer arasındaki fark kaçtır?$$A)20\;\;B)27\;\;C)31\;\;D)42\;\;E)62$$
\\
\textbf{Soru 2-} $x$,$y$ ve $z$ $0$'dan farklı gerçek sayılardır.$$\frac{1}{x}+\frac{1}{y}=\frac{1}{6},$$$$\frac{1}{y}+\frac{1}{z}=\frac{3}{2},$$$$\frac{1}{x}+\frac{1}{z}=\frac{1}{3}$$ olduğuna göre $$\frac{1}{x}+\frac{1}{y}+\frac{1}{z}$$ toplamı aşağıdakilerden hangisidir?$$A)-1\;\;B)1\;\;C)2\;\;D)4\;\;E)6$$
\\ \\ \\ \\
\textbf{Soru 3-}$$\systeme*{ax+9y-1=0,4x+ay-5=0}$$denkleminin çözüm kümesi boşküme ise $a$ gerçek sayısının negatif değeri aşağıdakilerden hangisidir?$$A)-36\;\;B)-6\;\;C)-3\;\;D)-2\;\;E)-1$$
\\
\textbf{Soru 4-} $a$ ve $b$ birer gerçek sayıdır. Buna göre $$(i)|ab|=|a||b|,$$ $$(ii)\left|\frac{a}{b}\right|=\frac{|a|}{|b|},$$ $$(iii)|a^b|=|a|^b,$$ $$(iv)|a-b|=|b-a|$$ifadelerinden hangileri daima doğrudur?$$A)i,ii\;\;B)i,iii\;\;C)i,iv;\;D)ii,iv\;\;E)i,ii,iv$$
\\
\textbf{Soru 5-} $\displaystyle\systeme*{\frac{1}{x}<\frac{2}{3},-\frac{x}{2}>x-10}$ eşitsizliklerini sağlayan kaç farklı $x$ tam sayısı vardır?$$A)1\;\;B)2\;\;C)3\;\;D)4\;\;E)5$$

\textbf{Soru 6-} $\displaystyle\frac{1}{9}<a<b<c<\frac{2}{9}$ sıralamasında birbirini izleyen sayılar arasındaki fark eşit olduğuna göre $\frac{c}{b}$ aşağıdakilerden hangisidir?$$A)\frac{8}{7}\;\;B)\frac{7}{6}\;\;C)\frac{6}{5}\;\;D)\frac{5}{4}\;\;E)\frac{4}{3}$$
\\
\textbf{Soru 7-} $x$ negatif bir gerçek sayı olmak üzere $$\frac{\sqrt{x^2}+\sqrt[3]{(-x)^3}+\sqrt[4]{(-x)^4}}{2x+\sqrt{(-x)^2}}$$işleminin sonucu aşağıdakilerden hangisidir?$$A)-1\;\;B)-\frac{1}{3}\;\;C)1\;\;D)\frac{1}{3}\;\;E)3$$
\\
\textbf{Soru 8-} $\displaystyle\systeme*{\frac{x}{2}+\frac{y}{3}=0,x-y=125}$ denklem sistemini sağlayan $x$ değeri kaçtır?$$A)-75\;\;B)-25\;\;C)15\;\;D)20\;\;E)50$$
\\
\textbf{Soru 9-} Ayşe ile Fatma'nın yaşları toplamı $45$'tir. Fatma Ayşe'nin yaşına geldiğinde Ayşe $36$ yaşında olacaktır. Buna göre Fatma'nın bugünkü yaşı kaçtır?$$A)12\;\;B)14\;\;C)15\;\;D)17\;\;E)18$$
\\
\textbf{Soru 10-}$$\systeme*{ax+9y+3=0,4x+ay+2=0}$$denklem sisteminin çözüm kümesi boşküme olduğuna göre $a$ kaçtır?$$A)-6\;\;B)-3\;\;C)0\;\;D)3\;\;E)6$$

\textbf{Soru 11-} Merve, sayı doğrusu üzerinde $-8$ ile $16$ noktaları arasına, bu aralığı eşit uzunlukta alt aralıktara bölecek şekilde kalemle işaretler koyuyor. Örneğin bu aralığı iki eşit aralığa bölmek için 4 noktasını aşağıdaki gibi işaretlemiştir:
\begin{center}
\begin{tikzpicture}
\draw[<->] (0,0) -- (6,0);
\draw (1,0) node{.};
\draw (5,0) node{.};
\draw (3,0) node{.};
\draw (1,0) node{.};
\node[] at (1,0.2) {-8};
\node[] at (5,0.2) {16};
\node[] at (3,0.2) {4};
\end{tikzpicture}
\end{center}
Merve, bu aralığı ayrı ayrı alt aralıklara bölüp işaretliyor. İşaretlediği noktalardan \underline{en az biri} bir tam sayının üzerine geldiğine göre Merve'nin sayı doğrusu üzerine koyduğu işaret sayısı aşağıdakilerden hangisi olabilir?
$$A)4\;\;B)6\;\;C)9\;\;D)11\;\;E)12$$

\section{Çözümler}

\textbf{Soru 1-} 12, $1\cdot12$, $2\cdot6$ ve $3\cdot4$ olarak yazılabileceğinden ve 18, $1\cdot18$, $2\cdot9$ ve $3\cdot6$ olarak yazılabileceğinden, $a+b+c$, en fazla $12+1+18=31$, en az $2+6+3=11$ olur ve $31-11=20$ olduğundan \textbf{cevap 20} olur.
\\ \\
\textbf{Soru 2-} Eşitlikleri toplarsak $$2\left(\frac{1}{x}+\frac{1}{y}+\frac{1}{z}\right)=\frac{12}{6}=2$$ buluruz yani \textbf{cevap 1} olur.
\\ \\
\textbf{Soru 3-} Çözüm kümesi boşküme ise eğimleri oranı birbirine eşit ve katsayıları oranından farklı olmalıdır çünkü öyle olursa düzlemde kesiştikleri bir nokta bulunamaz. Yani $$\frac{a}{4}=\frac{9}{a}\neq\frac{1}{5}$$ olmalıdır ve buradan $a^2=36\Rightarrow a=6,-6$ gelir. Yani, negatif değer sorulduğundan, \textbf{cevap -6} olur.
\\ \\
\textbf{Soru 4-} \textbf{(i)} $a$ ve $b$ ne olursa olsun, $|ab|$ ve $|a||b|$, $ab$'nin pozitif değerini ifade ettiği için eşitlerdir. Yani $(i)$ doğrudur. \textbf{(ii)} $(i)$'deki gibi, $t=b^{-1}$ dersek ifadeler $at$'nin pozitif değerine döner. Yani $(ii)$ doğrudur. \textbf{(iii)} $(i)$'den dolayı, $t_1=|a^{b-1}|$, $t_2=|a^{b-2}|$,$\cdots$,$t_b=|a^0|$ dersek, $t_1=|a|t_{2}=\cdots=|a|^{b-1}t_b=|a|^{b-1}\Rightarrow |a|t_1=|a^b|=|a|^b$ çıkar yani $(iii)$ doğrudur. \textbf{(iv)} İkisi de $a-b$'nin pozitif değerini ifade ettiği için aynıdır. Yani $(iv)$ doğrudur. Bunlardan yola çıkarsak \textbf{cevap i,ii,iii,iv} olur. Yani şıklar yanlış.
\\ \\
\textbf{Soru 5-} İlk eşitsizlikten $\frac{3}{2}<x$ çıkar ve ikinci denklemden $x<\frac{20}{3}$ çıkar. Yani $$\frac{3}{2}<x<\frac{20}{3}$$ olur ve aradaki sayılar 2,3,4,5 ve 6 olduğu için \textbf{cevap 5} olur.
\\ \\
\textbf{Soru 6-} Her sayı arasında $x$ fark varsa $\frac{1}{9}+4x=\frac{2}{9}$ olur ve buradan $x=\frac{1}{36}$ olur. Demek ki $c=\frac{1}{9}+3x=\frac{7}{36}$ ve $b=c-x=\frac{6}{36}=\frac{1}{6}$ olur ve $$\frac{c}{b}=\frac{7}{36}\cdot\frac{6}{1}=\frac{7}{6}$$ olur. Yani \textbf{cevap $\frac{7}{6}$} olur.
\\ \\
\textbf{Soru 7-} Köklü sayıların özelliklerini kullanarak $$\frac{(-x)+(-x)+(-x)}{x}=\frac{-3x}{x}=-3$$ çıkar. Yani \textbf{cevap -3} olur.
\\ \\
\textbf{Soru 8-} İkinci denklemden $y=x-125$ çıkar ve ilk denklemde yerine koyarsak $$\frac{x}{2}+\frac{x-125}{3}=0\Rightarrow \frac{x}{2}=\frac{125-x}{3}\Rightarrow 3x=250-2x$$ çıkar ve buradan $x=50$ olarak bulunur. Yani \textbf{cevap 50} olur.
\\ \\
\textbf{Soru 9-} Ayşe'nin yaşına $a$ ve Fatma'nın yaşına $f$ dersek elimizde $a+f=45$ ve $a-f=x$ ise $a+x=36$ olacağından $2a-f=36$ denklemleri var. Bunlardan $$a=45-f=\frac{36+f}{2}\Rightarrow 90-2f=36+f$$ çıkar ve buradan $f=18$ bulunu. Yani \textbf{cevap 18} olur.
\\ \\
\textbf{Soru 10-} Soru 3'teki gibi yaparsak $$\frac{a}{4}=\frac{9}{a}\neq\frac{3}{2}$$ olur ve buradan $a=-6$ gelir. Yani \textbf{cevap -6} olur.
\\ \\
\textbf{Soru 11-} Aralığın uzunluğu $8+16=24$ olur ve bu aralığı, $24=1\cdot24=2\cdot12=3\cdot8=4\cdot6$ olduğundan bu şekillerde eşit parçalayabiliriz. 11 nokta koyarsak 12 eş parçaya ayrılacağından dolayı ve şıklarda bunu sağlayan bir tek 11 olduğundan dolayı \textbf{cevap 11} olur.
\end{document}